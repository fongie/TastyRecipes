\documentclass[a4paper]{scrreprt}
\usepackage{fancyhdr}
\pagestyle{fancy}
\usepackage[english]{babel}
\usepackage[utf8]{inputenc}
\usepackage{graphicx}
\usepackage{url}
\usepackage{textcomp}
\usepackage{amsmath}
\usepackage{lastpage}
\usepackage{pgf}
\usepackage{wrapfig}
\usepackage{fancyvrb}
\usepackage{pdfpages}

\usepackage{etoolbox}
\makeatletter
\patchcmd{\scr@startchapter}{\if@openright\cleardoublepage\else\clearpage\fi}{}{}{}
\makeatother

\usepackage[colorlinks=true, linkcolor=violet]{hyperref} %links with \href{}
\usepackage[figure]{hypcap} %jump to img instead of caption

\newcommand{\code}[1]{\texttt{#1}}

% Create header and footer
\headheight 27pt
\pagestyle{fancyplain}
\lhead{\footnotesize{Applikationer för internet, ID1354}}
\chead{\footnotesize{Assignment 4 report}}
\rhead{}
\lfoot{}
\cfoot{\thepage\ (\pageref{LastPage})}
\rfoot{}

% Create title page
\title{Assignment 4}
\subtitle{Applikationer för internet, ID1354}
\author{Max Körlinge, korlinge@kth.se}
\date{\today}

\begin{document}

\maketitle

\tableofcontents %Generates the TOC
\clearpage

\chapter{Introduction}

This assignment is about using browser javascript, jQuery and Ajax calls. You were required to use AJAX for reading and writing comments (and I initially had a version of the assignment where you had to do login and logout aswell, which I have done). That was the mandatory task. One optional task was to do this using a framework for the viewmodel. I chose to do this optional task aiming for 1 bonus point, using the Knockout framework, as suggested in the assignment. There was an optional task to use Long Polling, which I did not do.

\chapter{Literature Study}

To complete the tasks I first studied the course lecture notes on Javascript and extensively studied the Knockout framework documentation.

\chapter{Method}

\section{Optional Task 1}

I implemented the viewmodel using separate javascript files for separate pages (i.e one for login page, one for recipe pages, etc). To keep up to date I used ES6 syntax javascript as much as possible, which is now available by default in modern browsers. This way, I tried both to keep the viewmodel object oriented (using ES6 classes), and to keep cohesion and encapsulation high.

For the server calls in fetching recipe comments and logging in/out I used jQuerys \code{\$.get} and \code{\$.post} methods. I changed the server parts to send the replies using PHPs \code{json\_encode} function, to get the replies in JSON format, easy to parse in javascript. To generate the comment divs in the view I used a foreach knockout binding to dynamically render the list of comments. The data sent from server is stored in the viewmodel's classes and not in HTML, they are then generated by the knockout framework.

To store the comments I use a knockout observable array called \code{comments} and on posting or getting with Ajax, I also add or remove the comment from the observable array, updating the page instantly in the client browser.

\chapter{Result}
\label{sec:result}

The git repository can be found \href{https://github.com/fongie/TastyRecipes/tree/assignment4}{here}.

\section{Optional Task 1}

The result is HTML pages that are completely free of data, and instead javascript classes contain all the data, from where it is output to the browser by the Knockout framework.

The viewmodel files are \href{https://github.com/fongie/TastyRecipes/tree/assignment4/src/view/viewmodel}{here}. As an example, you can see \href{https://github.com/fongie/TastyRecipes/tree/assignment4/src/view/viewmodel/commentSection}{the commentSection js file}, for how comments are fetched, posted, and deleted using ajax get and post requests. The data bindings for knockout, and the only HTML shown in the DOM, is in \href{https://github.com/fongie/TastyRecipes/tree/assignment4/src/view/parts/comments.php}{this file}. You can see that there is no data in the HTML file, and by the \code{data-bind} attributes that Knockout framework is used. The server sends only data as a response to the ajax requests. All such requests can be found in the \href{https://github.com/fongie/TastyRecipes/tree/assignment4/src/view/requests}{requests directory}, where you can easily see that what is returned by the server is always \href{https://github.com/fongie/TastyRecipes/blob/b3511142454b98a722c9f1fa76f3680c740ec1b4/src/view/requests/get\_comments.php#L8}{json data}.

Recipe comments are stored in the knockout observable array \code{comments} belonging to the {commentSection} class. Posting and deleting comments instantly shows up on the page, since the methods in the viewmodel both post to the server and at the same time update the viewmodel's observable array, as seen \href{https://github.com/fongie/TastyRecipes/blob/d5a045f5de354d46aa2ed07f5e2d84885910bad4/src/view/viewmodel/commentSection.js#L45}{here}. The post and delete methods were a bit tricky. For the post method I had to add to the \code{post\_comment} file, and to several of the underlying server files, some code that returns what ID the comment last added has, to be able to delete it afterwards. For deleting comments, to remove the correct comment from the observable array I use the \code{remove()} method to find the correct comment to remove.

\chapter{Discussion}

Knockout had some quirks when it came to applying bindings which stumped my attempt to divide the viewmodel into separate files at first, telling me it couldn't bind to the same elements twice. I did make it work in the end, and this way all javascript doesn't have to be loaded for every page, only where it is necessery. This, I would think, is a viewmodel where data is separate from the HTML.

It has been interesting to work in Javascript and with Ajax since it clearly shows the asynchronous nature of javascript, you cannot count on things happening in the order you predicted. It was only at the end that I managed to find a good way to instantly update the page on comment post and delete without reloading the page - modifying the knockout observable array in the callback function to the ajax responses. Ajax, as well as the Knockout framework, is thus clearly used for reading, writing, and deleting comments, as required by the assignment.

There was never any HTML sent as a response to requests to the servers, so fulfilling that requirement was never an issue. It was simply to use \code{json\_encode} on the server side to receive the data for javascript.

\chapter{Comments About the Course}

I spent about 30 hours on this assignment.
This assignment looked easy at first since it was only one (or two) tasks, but it turned out to be somewhat difficult to manouver knockout at the same time as managing the ajax requests.
This is the last assignment for the course and I think that we have step by step learned a good way to structure a web application, using several web technologies I did not know previously.

\end{document}
